\documentclass[11pt,a4paper]{article}
\usepackage{a4wide}
\usepackage{graphicx}
\usepackage{epsfig}
\usepackage{amsmath}
\usepackage{amsfonts}
\usepackage{amssymb}
\usepackage{ngerman}
\usepackage[T1]{fontenc}
\usepackage[utf8]{inputenc}
\inputencoding{latin1}
\usepackage{multirow}
\usepackage{hyperref}
\usepackage{float}
\usepackage{url}
\usepackage{csvsimple}
\usepackage{caption} 

\begin{document}


\section{Schema fuer Jacobi}
1) Datenaufteilung der Matrix auf die einzelnen Prozesse:\\
Die N Prozesse bekommen gleich viele Zeilen der Gesamtmatrix als Teilmatrix
zugewiesen. Die jeweilige Anzahl an Zeilen Nz erh�lt man, indem man die Anzahl der Zeilen 
der Gesamtmatrix  ganzzahlig durch numproc teilt. Geht ein Teilen durch Numproc nicht auf wird durch (numProc-1) geteilt. Bleibt kein Rest, werden die Nz wieder durch numProc geteilt. Alle Prozesse bekommen dann Nz/numProc Zeilen, bis auf den letzten (Nz/numProc)+modulo(Nz/numProc). Bleibt bei Nz/numProc ein Rest bekommen alle Prozesse Nz/numProc  Zeilen und der letzte den Rest.  \\
Bei der Verteilung muss au�erdem Beachtet werden, dass die einzelnen Prozesse zur Berechnung 
des Abtaststerns jeweils die Randwerte von anderen Prozessen ben�tigen. Die Teilmatrizen 2 bis 
N-1 ben�tigen deshalb zus�tzlich jeweils eine Zeile unten und oben. Die Teilmatrix 1 ben�tigt nur 
eine Zeile unterhalb, die Teilmatrix N nur eine Zeile oberhalb. \\
Bei der Aufteilung mit \textit{MPI\_SCATTERV} werden die Teilmatrizen deshalb so aufgeteilt, dass sich sich 
jeweils in einer Zeile �berlappen. So liegen f�r den ersten Iterationsschritt auch bereits 
alle ben�tigten Randwerte vor.  \\

2) Parallelisierungsschema \\
Nach der Aufteilung mit \textit{MPI\_SCATTERV} haben alle Teilmatrizen alle ben�tigten Randwerte (siehe 1)). 
Der erste Schritt in jedem Teilprozess und jedem Iterationsschritt 
ist deshalb die Berechnung des Abtaststerns f�r jedes Matrixelement (au�er f�r die Randwerte). 
Danach m�ssen die Ergebnisse, die von anderen Prozessen widerum als Randwerte ben�tigt werden 
kommuniziert werden: \\
Die Prozesse i=1 bis i=N-1 senden ihre zweitunterste Zeile (nicht die Zeile mit den Randwerten, sondern die
dar�ber!) an den darauffolgenden Prozess (i+1). Der Prozess i+1 ersetzt seine oberste Zeile mit der empfangenen. \\
Die Prozesse i=2 bis i=N senden ihre zweitoberste Zeile an den vorherigen Prozess (i-1). Der Prozess 
i-1 erstetzt seine unterste Zeile mit der empfangenen. \\
F�r einen einzelnen Prozess i bedeutet das: er f�hrt zuerst die Berechnung durch. Danach sendet er
seine zweitunterste Zeile an den n�chsten Prozess und seine zweitoberste an den vorherigen Prozess. 
Anschlie�end empf�ngt er jeweils eine Zeile vom vorherigen und nachfolgenden Prozess und ersetzt seine 
erste und letzte Zeile durch die empfangenen Zeilen. Bei der ersten (letzten)
Teilmatrix ist zu beachten, dass diese keine neuen Randwerte f�r ihre erste (letzte) Zeile empf�ngt.
In diesen Zeilen gelten die konstanten Randbedingungen, die zu Beginn festgesetzt werden.  \\

3) Abbruchkriterium\\
bei fester Iterationszahl:\\
Dieses Abbruchkriterium ist einfach zu implementieren. Die Schleife zur Berechnung und Kommunikation
wird genauso oft durchgef�hrt, wie die Anzahl an Iterationen ist. \\
bei Iteration nach Genauigkeit: \\
jeder Prozess setzt nach jedem Iterationsschritt eine logical Variable auf 0 oder 1. 
1 bedeutet, dass das Abbruchkriterium in diesem Prozess erf�llt ist, 0 bedeutet dass es nicht erf�llt 
ist. Diese Variable wird an den Masterprozess versendet, der mit MPI\_REDUCE alle logicals der
Teilprozesse empf�ngt und das Minimum berechnet. Ist das Minimum 0, ist das Abbruchkriterium noch 
nicht in jedem Teilprozess erf�llt und es muss weiter iteriert werden. Ist das Minimum 1, kann
die Iteration gestoppt werden. Falls dies der Fall ist, sendet der Master mit MPI\_BROADCAST eine logical
Variable an alle Teilprozesse, sodass diese 'wissen', dass die Iteration gestoppt werden muss.   \\

\section{Gauss-Seidel}
Die Aufteilung der Matrix geschieht genauso wie bei Jacobi.
Verfahren aus Sicht eines einzelnen Prozesses:\\

Halolinien sind Randlinien(Randbedingungen - oberste und unterste Linie einer Teilmatrix)\\
1. Empfangen der oberen Haloline vom vorherigen Prozess \\
2. Berechnung der ersten Zeile der Teilmatrix\\
3. Versenden dieser Zeile an den vorherigen Prozess\\
4. Berechnung der restlichen Zeilen der Teilmatrix\\
5. Versenden der letzten berechneten Zeile an den n�chsten Prozess \\
6. Empfangen der unteren Haloline vom n�chsten Prozess\\
Besonderheiten:\\
Beim nullten Prozesses entfallen die Schritte 1-3, beim letzten Prozesses entfallen die Schritte 5-6.\\
Abbruchkriterium nach Iteration:\\
Hier gibt es keine besonderen Schwierigkeiten. Die Schleife wird von jedem Prozess bis zur gew�nschten
Iterationszahl durchlaufen. Wie in Aufgabe 7A f�hrt der Masterprozess anschlie�end mit MPI\_GATHERV die 
Teilmatrizen wieder zusammen.\\

Abbruchkriterium nach Genauigkeit:\\
Zun�chst wird wie in Aufgabe 7b gepr�ft, wann alle Prozesse die gew�nschte Genauigkeit erreicht haben. 
Beim Gauss-Seidel Verfahren besteht nun die Schwierigkeit darin, dass sich die einzelnen Prozesse in 
verschiedenen Iterationen befinden. Der nullte Prozess hat dabei immer die gr��te Anzahl an Iterationen 
durchlaufen, da der nachfolgende Prozess immer auf die letzte berechnete Zeile des vorherigen warten muss. 
Mit MPI\_(I)REDUCE �berpr�ft der nullte Prozess, die Abbruchkriterien aller Prozesse. Sobald alle die 
gew�nschte Genauigkeit erreicht haben, h�rt der nullte Prozess auf zu rechnen und alle anderen Prozesse 
rechnen um Rank Iterationen weiter. So wird sichergestellt, dass alle Prozesse die gleiche Anzahl an 
Iterationen durchlaufen haben. Mit einem MPI\_REDUCE einer weiteren logical Variable, welches vom nullten 
Prozess ausgef�hrt wird, stellt dieser fest ob alle Prozesse fertig sind. Dann werden mit MPI\_GATHERV die 
Teilmatrizen durch den nullten Prozess zusammengef�hrt.

\end{document}